\section{项目经历}
%-------------------------------------------------------
\datedsubsection{\textbf{MiniC 编辑器} \href{https://github.com/yanshengjia/nucleon}{[\underline{Link}]}}{2016.10 --\ 2016.12}
\textit{独立开发}
\begin{itemize}
  \item 用 Python 和 Qt 开发了一个有语法高亮功能的 MiniC 编辑器
  \item 整合了这个编辑器和 MiniC 编译器以检测代码错误
\end{itemize}

%-------------------------------------------------------
\datedsubsection{\textbf{MiniC 编译器} \href{https://github.com/seucs/compiler}{[\underline{Link}]}}{2016.05 --\ 2016.06}
\textit{核心成员}
\begin{itemize}
  \item 用 Python 实现了 ``正则表达式 -> NFA'' 转换器
  \item 用 Python 实现了 LR(1) 分析器和相应的语义动作
\end{itemize}

%-------------------------------------------------------
\datedsubsection{\textbf{人工智能课程项目} \href{https://github.com/yanshengjia/artificial-intelligence}{[\underline{Link}]}}{2016.03 --\ 2016.06}
\textit{独立开发}
\begin{itemize}
  \item 使用了人工神经网络进行人脸识别
  \item 用 Matlab 实现了遗传算法来解决函数最值问题
  \item 用 C++ 实现了 A* 算法来解决24数码问题
  \item 用 C++ 实现 QS4 算法来解决百万皇后问题
\end{itemize}

%-------------------------------------------------------
\datedsubsection{\textbf{实体链接者} \href{https://github.com/acmom/entity-linker}{[\underline{Link}]}}{2016.03 --\ 2016.04}
\textit{组长}
\begin{itemize}
  \item 研究并比较了多种实体链接算法
  \item 开发了一个实体链接系统, 能够将 Web 表格中的指称链接到 Wikipedia 中的参考实体
\end{itemize}

%-------------------------------------------------------
\datedsubsection{\textbf{My Minecraft} \href{https://github.com/seucs/my-minecraft}{[\underline{Link}]}}{2015.09 --\ 2016.01}
\textit{组长}
\begin{itemize}
  \item 用 OpenGL 开发了一个类似于 Minecraft 的小型 3D 游戏
\end{itemize}

%-------------------------------------------------------
% \datedsubsection{\textbf{微博爬虫} \href{https://github.com/yanshengjia/crawler/tree/master/weiboCrawler}{[\underline{Link}]}}{2015.11 --\ 2015.11}
% \textit{独立开发}
% \begin{itemize}
%   \item 开发了一个多线程爬虫来从微博上爬取数据
% \end{itemize}

%-------------------------------------------------------
\datedsubsection{\textbf{虚拟校园} \href{https://github.com/acmom/vcampus}{[\underline{Link}]}}{2015.09 --\ 2015.10}
\textit{核心成员}
\begin{itemize}
  \item 开发了一个 Java 软件来管理学生信息
  \item 增加了一些额外功能, 比如在线聊天, 数字图书馆, 网上商店等
\end{itemize}